\renewcommand*{\mypath}{trovaintruso1}%
\graphicspath{ {\mypath/images/} }

\doctitle{Paolo Ferraris}{paolo2.ferraris@mail.polimi.it}
\localtoc

\section{Manuale Sviluppatore}

\subsection{Descrizione delle classi}

\noindent Per prima cosa vengono descritte le classi principali, partendo dal modello che definisce il gioco, in seguito vengono descritte le classi \code{Helper} che permettono la gestione della logica, e infine le Activity utilizzate.

\subsubsection{Classi del modello dati}

\noindent Di seguito sono descritte le classi che gestiscono il modello del gioco, rappresentate nel \textit{class diagram} della Figura \ref{fig:class_diagram}:

\begin{figure}
	\centering
	\includegraphics[width=13.5cm]{trova_l_intruso_class_diagram}
	\caption{Class Diagram}
	\label{fig:class_diagram}
\end{figure}
\begin{description}
\item \textbf{Game} \`{e} la classe che gestisce la partita, i principali metodi sono:
\begin{itemize}
\item \code{initialize()}: metodo che inizializza la sessione di gioco, creando i livelli e assegnandone gli oggetti da visualizzare.
\item \code{initializeMultiplayerSession(boolean guest)}: metodo che stabilisce i turni di gioco nelle partite multiplayer, la variabile guest indica se il dispositivo \`{e} quello che \`{e} stato invitato a giocare.
\item \code{goToNextScreen()}: metodo che attiva la schermata successiva e restituisce true. Nel caso ci si trovi nell'ultimo livello restituisce false.
\item \code{fillScreens()}: riempie le schermate con le immagini in base alla categoria e al numero di oggetti da visualizzare.
\item \code{CreateGameDescription(Context context)}: crea la descrizione della partita da mandare via mail.
\item \code{restart()}: riavvia il gioco con le stesse impostazioni.
\end{itemize}
\item \textbf{Settings} Classe che gestisce le impostazioni di gioco, possiede le seguenti propriet\`{a}:
\begin{itemize}
\item \code{\_singlePlayer(Boolean)}: indica la modalit\`{a} di gioco.
\item \code{\_numOfObjects(int)}: indica il numero di immagini visualizzate per livello.
\item \code{\_numOfScreens(int)}: indica il numero di livelli impostati.
\item \code{\_category(Category)}: indica la categoria di immagini selezionata.
\end{itemize}

\item \textbf{Screen}
Classe che rappresenta il livello di gioco, possiede come propriet\`{a} il tempo di inzio livello, il tempo di fine, una variabile booleana che indica se \`{e} il turno del giocatore corrente, il numero di errori e un array contenente le immagini da visualizzare. I principali metodi sono:
\begin{itemize}
\item \code{initialize()}: inizializza il livello, inserendo le immagini.
\item \code{init()}: azzera le variabili del livello, ossia numero di errori, tempo di gioco.
\item \code{start()}: avvia il gioco su questo livello, settando la variabile \code{\_start\_time}.
\item \code{error()}: incrementa il conteggio degli errori di uno.
\item \code{completed()}: metodo evocato quando il livello \`{e} completato, setta la variabile \code{\_end\_time}.
\item \code{getScreenTime()}: restituisce il tempo di gioco.
\end{itemize}
\item \textbf{Element}
\noindent Classe che rappresenta un immagine, possiede due attributi, il nome dell'immagine presente nella cartella \code{drawable} e un valore booleano che indica se \`{e} l'elemento target.
\item \textbf{Category}
Classe che rappresenta una categoria di immagini, possiede come propriet\`{a} il nome della categoria, un campo booleano che indica se si tratta di una categoria ``Casuale'' e una lista di oggetti CategoryGroups relativi alla categoria. Il costruttore accetta un JSONObject contenente la definizione della categoria e da esso inizializza l'oggetto.
\item \textbf{CategoryGroup}
Classe che rappresenta una gruppo di immagini che deve essere rappresentato in una schermata. Definisce l'elemento target e i restanti elementi.
\end{description}

\subsubsection{Classi Helper}

\begin{description}
\item \textbf{CategoryHelper} Gestisce il caricamento delle categorie di immagini dal file \code{.json} dove sono definite.
\item \textbf{GameHelper} Gestisce le meccaniche di gioco e la comunicazione tra i dispositivi in multiplayer. I principali metodi di questa classe sono:
\begin{itemize}
\item \code{onMainActivityCreate()}: metodo utilizzato all'avvio dell'activity principale, avvia il riconoscimento dei dispositivi sul network (per le partite in multiplayer) e carica le impostazioni di gioco precedentemente salvate.
\item \code{onMainActivityDestroy()}: chiamato quando l'activity principale viene distrutta, viene terminato il servizio di discovery sul network e viene chiusa la connessione (se attiva).
\item \code{registerCurrentActivity(Activity activity)}: metodo chiamato nell'onCreate di ogni activity, registra l'activity mostrata a schermo per gestire al meglio le comunicazioni tra i dispositivi in multiplayer e la visualizzazione di avvisi o dei feedback di gioco.
\item \code{startGame()}: Questo metodo inizializza il gioco, viene chiamato il metodo \code{Game.initialize()} e in seguito se il gioco \`{e} in SinglePlayer viene caricata l'activity Screen nella quale viene mostrato il primo livello, altrimenti si passa all'activity MultiPlayerDiscoveryActivity per invitare un giocatore.
\item \code{quitGame()}: questo metodo serve per terminare una partita, e nel caso di partite multiplayer viene chiusa anche la connessione.
\item \code{nextScreen()}: metodo che serve per passare al livello successivo, in caso di partite multiplayer viene inviato l'analogo messaggio all'altro dispositivo.
\end{itemize}
\item \textbf{MultiPlayerServiceHelper} Gestisce il riconoscimento dei dispositivi che eseguono il gioco sulla rete per permettere l'avvio di partite multiplayer.
\item \textbf{ConnectionHelper} Il multiplayer \`{e} implementato usando i socket Java, questa classe gestisce la connessione, l'invio di messaggi sul network e inizializza gli oggetti \code{ServerHelper} e \code{ClientHelper}.
\item \textbf{ServerHelper} Gestisce il lato server della connessione, avviando un \code{SocketServer} thread che si occupa dell'assegnamento di una connessione al socket client appena arriva una richiesta al server.
\item \textbf{ClientHelper} Questa classe istanzia due thread, un thread che si occupa della ricezione dei messaggi in arrivo al socket utilizzato per la connessione, e uno che si occupa di inviare i messaggi al dispositivo col quale si sta giocando.
\end{description}

\subsubsection{Gestione della comunicazione in partite multiplayer}
\noindent Per la gestione delle comunicazioni tra dispositivi nelle partite multiplayer \`{e} stata creata una classe \code{GameMessage}, la quale definisce una propriet\`{a} di tipo \code{Serializable}, che rappresenta il contenuto del messaggio, e una propriet\`{a} chiamata \code{Type} un enum che indica il tipo di messaggio, che pu\`{o} assumere i seguenti valori:
\begin{itemize}
\item \code{ConnectionRequest}: i messaggi di questo tipo non hanno contenuto e servono ad inoltrare a un dispositivo una richiesta di connessione per avviare una partita;
\item \code{ConnectionAccepted}: questo messaggio viene inviato dal client al server quando il giocatore accetta di giocare;
\item \code{ConnectionClosed}: questo messaggio viene inviato prima di chiudere la connessione;
\item \code{SendGame}: con questo messaggio viene inviata la struttura del gioco al dispositivo invitato a giocare;
\item \code{SendGameAck}: questo messaggio conferma la ricezione del messaggio precedente da parte del client;
\item \code{StartGame}: questo messaggio viene inviato al dispositivo client per comunicargli di avviare il gioco, alla sua ricezione il client avvia il gioco e manda il messaggio di ack al server;
\item \code{StartGameAck}: questo messaggio conferma la ricezione del messaggio precedente, alla sua ricezione il dispositivo server a sua volta avvia il gioco;
\item \code{ElementPressed}: con questo messaggio viene inviato l'indice dell'elemento premuto all'altro dispositivo in modo da visualizzare il feedback;
\item \code{NextScreen}: con questo messaggio si comunica al dispositivo avversario che il giocatore ha premuto il pulsante per cambiare schermata.
\end{itemize}

\subsubsection{Activity}

\noindent Di seguito vengono descritte le principali Activity presenti nell'app:

\begin{description}
\item \textbf{MainActivity}: \`{e} l'activity che si presenta all'avvio del gioco, inizializza la propriet\`{a} globale \code{GameHelper} e permette di avviare il gioco.
\item \textbf{ScreenActivity}: l'activity che gestisce il livello di gioco, all'avvio vengono caricate le immagini e mostrate a schermo, in seguito si attende l'input dell'utente e si visualizzano a schermo i feedback.
\item \textbf{ResultsActivity}: l'activity che viene mostrata alla fine del gioco, mostra i risultati dei livelli e permette di riavviare il gioco oppure tornare alla MainActivity.
\item \textbf{MultiPlayerDiscoveryActivity}: in questa activity vengono mostrati gli altri giocatori e si pu\`{o} iniziare una partita multiplayer.
\item \textbf{SettingsActivity} e \textbf{ConfigDeviceActivity} permettono la configurazione delle impostazioni della partita e del dispositivo.
\end{description}

\subsection{Modello della sessione di gioco}

\noindent
Nella Figura \ref{fig:activity_diagram} \`{e} riportato l'activity diagram che schematizza lo svolgimento del gioco.

\begin{figure}
	\centering
    \includegraphics[width=13.5cm]{trova_l_intruso_android}
	\caption{Class Diagram}
	\label{fig:activity_diagram}
\end{figure}

%\subsubsection{Inizializzazione di una partita mutliplayer}

%\noindent Il sequence diagram



\section{Manuale Utente}
Dalla schermata principale (figura \ref{fig:intruso1_main}), toccando l'icona grigia in alto a sinistra dello schermo, si può accedere alla schermata delle impostazioni (figura \ref{fig:intruso1_settings}) da cui si possono impostare i parametri di gioco. Per impostare l'indirizzo e-mail per l'invio dei dati, invece, bisognerà toccare ``Configura il dispositivo'' per accedere alla schermata in figura \ref{fig:intruso1_devicesettings}.
\screenshot{intruso1_main}{La schermata principale}
\screenshot{intruso1_settings}{La schermata di impostazioni}
\screenshot{intruso1_devicesettings}{La schermata di impostazioni per l'invio dei dati di gioco}

Sempre dalla schermata iniziale si può dare inizio al gioco sia in modalità multi-giocatore, che singolo. Se si sceglie la prima, allora si verrà rimandati alla schermata di ricerca di altri giocatori connessi alla medesima rete \textit{wi-fi} (figura \ref{fig:intruso1_multi}).

\screenshot{intruso1_multi}{La schermata di ricerca di altri giocatori}

Una volta iniziata la partita (figura \ref{fig:intruso1_game}), bisognerà toccare gli intrusi per concluderla e vincere.

\screenshot{intruso1_game}{La schermata di gioco}
